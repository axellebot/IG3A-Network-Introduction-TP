\chapter{TP3 : Outil de simulation de réseaux : NS-2}
    On commence par réaliser le réseau demandé. Grâce au TP précédent, cela se fait rapidement, et selon les spécifications du sujet. On organise le code de la même façon que dans le TP précédent, pour plus de clareté. Le but de ce TP est de comparer les différences en termes de ... ... de deux types de routage, à Vecteur de distance et à Etat de lien. En ne changeant que ce facteur lors de 2 simulations, on pourra, à l'aide des outils fournis dans nam, facilement distinguer ce qui advient des paquets en transit pendant la perte de connexion, et comment les différents noeuds du réseau communiquent afin de réaliser le routage.
